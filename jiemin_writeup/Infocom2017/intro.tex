%!TEX root = r-gather.tex

%\pagebreak

\section{Introduction}

We study the problem of clustering mobile nodes into meaningful, highly location-dependent clusters.  In order to quantify ``meaningful'' clusters, we establish a lower bound, $r$, on the size of clusters.  The resulting {\em $r$-gather} clustering problem is formally stated:
Given a set of $n$ points $P = \{p_1, p_2, \dots, p_n\}$ in Euclidean space and a value $r$, cluster the points into groups of at least $r$ points each such that the largest diameter of the clusters is minimized. We consider two popular notions of ``diameter'' of a cluster: the usual notion of {\em diameter} of a point set (the maximum distance between two points of the set), and the diameter of the minimum enclosing ball (MEB) of the set.

%\medskip\noindent\textbf{Motivations.} 
\subsection{Motivation}
One motivation of this version of clustering arises in location privacy in wireless networking. With the ubiquitous use of GPS receivers on mobile devices, it is now common practice that the locations of these mobile devices are recorded and collected. This raises privacy concerns as location information is sensitive and can be used to identify the user of the devices~\cite{blumberg2009}. This problem is more challenging when location is part of the input in location-based queries, for example, finding the coffee shop nearest to the user, querying traffic situations, and security-related applications such as reporting suspicious behaviors. In these settings, the user submits a query to location-based services (LBS) through a mobile device. An adversary that compromises the LBS server can infer private information about the user. In this setting, protection of privacy is characterized into two different yet related types: \emph{query privacy}, e.g., whether an adversary can identify the user who issued the query (i.e., associate user IDs with queries), and \emph{location privacy}, e.g., how much an adversary can learn regarding location of a user. %Most privacy protection algorithm use location perturbation and obfuscation methods. 
A nice survey on this topic can be found in~\cite{Chow:2011:TPL:2031331.2031335,6155874}.

For query privacy, the most common approach is to use the $k$-anonymity measure~\cite{Sweeney02}, which says that the locations are grouped into clusters, each of at least $k$ points. 
In previous work~\cite{Gruteser:2003:AUL:1066116.1189037,Mokbel:2006:NCQ:1182635.1164193} ``cloaking boxes'' were used to group spatio-temporal user queries into a box with at least $k$ queries and then submit the box (instead of the queries themselves) to the LBS server. In this way, the query sender is indistinguishable from the $k-1$ other query users in the same box. It is ideal to group queries from similar locations into the same box.
%One common practice in the treatment of location data is to adopt the $k$-anonymity criterion~\cite{Sweeney02}, which says that the locations are grouped into clusters, each of at least $k$ points. The cluster is used to replace individual locations, so that any one user is not differentiated from the $k-1$ others. 
Towards this goal, one objective to minimize the diameter of the clusters, since this yields location data with the best possible accuracy, while not intruding on user privacy.  This is precisely the $r$-gather problem~\cite{Aggarwal06achievinganonymity}.

%\pagebreak

%Protecting location privacy was brought to the attention of the research community a few years ago~\cite{blumberg2009}. A number of work has been conducted to protect the privacy of a \emph{single} location, when the user submits a query to location-based services (LBS) on mobile phones. An adversary that compromises the LBS server can infer private information about the user. In this setting, protection of privacy is characterized into two different yet related types: \emph{query privacy}, e.g., whether an adversary can identify the user who issued the query, and \emph{location privacy}, e.g., whether a user can be accurately located. Most privacy protection algorithm use location perturbation and obfuscation methods. A nice survey can be found in~\cite{Chow:2011:TPL:2031331.2031335,6155874}.

%For location privacy, one common metric used is location entropy, which characterizes the uncertainty of the location information an adversary can extract from LBS queries. Most methods use perturbations of the true locations to confuse the adversary (for example, see~\cite{1607571,4497445,Hong:2004:APU:990064.990087}). Notice that the reported location cannot be too far away from the true location as otherwise the location based query will become useless. There is a tradeoff of location privacy versus query utility.

Apart from protecting the privacy of a single snapshot, which is formulated by the static $r$-gather problem, it is natural to consider the dynamic setting for mobile users. 
%Protecting the privacy of the trajectory data is not investigated as much. 
%This is partly because the unique geometric structures in the trajectory data that call  for novel geometric algorithms. Previous work is along the following directions. %A small number of prior work was available.
%In~\cite{1506394} and~\cite{You:2007:PMT:1548880.1548998}, dummies with movement patterns similar to those of real users are generated and sent to the LBS to confuse the adversary. But two problems remain: the adversary may differentiate real users from dummies after long time tracking, and the added noises alter the properties of the data set and consequently decrease the data utility. %In~\cite{You:2007:PMT:1548880.1548998}, the original trajectories are first analyzed in three metrics about the `degree of privacy protection': snapshot disclosure, trajectory disclosure and distance deviation. Then dummy trajectories are added until the final set satisfies the requirement. One issue to be aware of for the inclusion of dummies is how much the added noises alter the properties of the data set and consequently decrease the data utility.
When mobile phones continuously issue location-based services, temporal spatial cloaking boxes are created~\cite{Chow:2007:EPC:1784462.1784477,Xu:2007:LAC:1341012.1341062,4509698}. The challenge is that the cloaking box may become huge after a long time period, leading to high computational cost and low query accuracy. Clearly there is a tradeoff between the quality of the cluster size and the number of re-clusterings we want to afford. Formulating this problem using the $r$-gather framework, we may ask the follow question: Given a parameter $k$ specifying the maximum number of times we can recluster over the time period of interest, when and how should re-clustering be done so that the maximum cluster diameter is minimized, while satisfying the $r$-gather constraint that each cluster have at least $r$ elements?

Further, the $r$-gather problem has been investigated in centralized settings, when all information is gathered at a single location. In mobile applications, it is desirable to  develop distributed algorithms, with local operations and computations distributed among the nodes. 

Besides the natural connection to location privacy issues, the $r$-gather problem is a natural and useful variaent of mobile clustering in general. Many mobile applications rely on grouping the mobile nodes into clusters for management purposes. These clusters need to be geographically coherent and for a cluster to be meaningful, it is natural to require each cluster to have a minimum number of members. Clustering mobile nodes has been studied a lot in the past, for example in~\cite{593002,lin97adaptive,chen99clustering,basu01mobility,Basagni99distributed,gghzz-dmc-03}. But none of the previous work strictly enforce a lower bound on the cluster cardinality. 

%A different idea~\cite{1186725} introduces `mixed-zones', a geometric region such that users entering the zone change to a pseudonym and do not send LBS queries. If the mixed zone has $k$ users, the probability of each user leaving the zone along each exit is the same and each user spends a random duration of time inside the zone, the set of users are $k$-anonymized. Such mixed zones are created both as geographical boxes or segments along road network~\cite{LCA-CONF-2007-016,5767898}. This scheme requires hardware support to perform pseudonym change. It also imposes limitations to user queries within the zone.

%In the case of protecting privacy during publication of trajectories to a third party, two ideas have been mainly used: clustering based~\cite{4497446,Gao:2014:BTP:2567003.2567226} and generalization based~\cite{Nergiz:2009:TTA:1556406.1556410,Monreale:2010:MDA:1824401.1824403}. A set of trajectories have two distinct structures -- the coordinates of data points in each temporal snapshot, and how these data points are linked to form trajectory paths. The clustering based approach alters the former while the generalization based~\cite{Nergiz:2009:TTA:1556406.1556410,Monreale:2010:MDA:1824401.1824403} algorithm alters the later. In clustering based approach, the trajectory data is clustered into groups of $k$ co-located trajectories within the same time period to form a $k$-anonymized aggregated trajectory. The limitation of this approach is the strong requirement that such clusters can be found in the data set. In addition, many applications require atomic trajectories and are not ready to take aggregated trajectories. In generalization based approach, each trajectory is first transformed into a sequence of $k$-anonymized regions. Then for each trajectory, the algorithm uniformly select $k$ atomic points in each anonymized region and link a unique atomic point from each region to reconstruct $k$ trajectories. Performance of this approach can be evaluated in two aspects -- how big the anonymized regions are and how much the altered data change the utility of this data set. The scheme by \cite{Chen:2013:PTD:2442161.2442241} can be considered as a mixture of clustering and generalization. Besides, a few other papers make separate assumptions on the power of the adversary and consider different metrics~\cite{Terrovitis:2008:PPP:1397755.1397842,Yarovoy:2009:AMO:1516360.1516370}.

%A number of work has been done using differential privacy~\cite{Dwork:2011:FFP:1866739.1866758,Blum:2008:LTA:1374376.1374464,Dwork:2009:CDP:1536414.1536467,Xiao:2011:DPV:2006855.2007020,Xiao:2010:DPD:1889159.1889173,Jiang:2013:PTD:2484838.2484846,Hu:2010:PLD:1806907.1806910}. Most of the work focus on relational databases. Trajectory data has strong sequentiality. The methods for relational databases typically face scalability issues in the context of trajectory data~\cite{Mohammed:2011:DPD:2020408.2020487,journals/pvldb/ChenMFDX11}.

%!TEX root = r-gather.tex

\subsection{Related Work}

The $r$-gather problem has been studied for instances in general metric spaces.  Aggarwal et al.~\cite{Aggarwal06achievinganonymity} give a $2$-approximation algorithm and show that, for $r>6$, it is NP-hard to approximate with an approximation ratio better than $2$.  The approximation algorithm first guesses the optimal diameter, then greedily selects clusters with twice the diameter; finally, a flow algorithm is used to assign at least $r$ points to each cluster.  This procedure is repeated until a good choice of diamter is found.  Note that this solution only selects input points as cluster centers.

Armon \cite{armon2011min} extended the result of Aggarwal et al. proving that, for $r>2$, it is NP-hard to approximate with a ratio better than $2$ for the case of general metric spaces.  Armon also considers a generalization of the $r$-gather clustering problem, called the {\em $r$-gathering problem}, which also considers a given set of potential cluster centers (potential ``facility locations''), each having a fixed set-up cost that is included in the objective function. Armon provides a $3$-approximation for the min-max $r$-gathering problem and proves that it is NP-hard to obtain a better approximation factor.  Additional results include various approximation algorithms for the min-max $r$-gathering problem with a {\em proximity requirement} that each point be assigned to its nearest cluster center.

For the case $r = 2$, both \cite{anshelevich2011terminal} and \cite{shalita2010efficient} provide polynomial-time exact algorithms.  Shalita and Zwick's \cite{shalita2010efficient} algorithm runs in $O(mn)$ time, for a graph with $n$ nodes and $m$ edges.

All of these prior algorithms were for the centralized setting, in general metrics; we are not aware of results in the distribtued and dynamic settings, or of results that exploit geometric structure. 

%% Joe added this discussion after submission to INFOCOM:
A related problem is that of the discrete disk cover: Given a set $S$ of points in a metric space, and a collection ${\cal B}$ of disks whose union covers $S$, find a smallest cardinality subset of ${\cal B}$ that covers $S$.  As a set cover instance, it is well known that the greedy algorithm yields a logarithmic factor approximation.  %%% xxx add a citation?  CLRS?
In geometric instances, e.g., in low-dimensional Euclidean space, it is known that local search yields a PTAS.  %% xxx cite Mustafa and Ray, and newer papers??
Since the number of combinatorially distinct disks is polynomial, in any fixed dimension Euclidean space, one could specify ${\cal B}$ to be the set of all (minimal) disks that contain at least $r$ points of $S$ and that have diameter at most some parameter $D$.  Minimizing the number of such disks in a cover of $S$, then, has a PTAS, from known results on discrete disk cover.  However, this problem is not the same as the $r$-gather cluster problem, since it is not partitioning the points of $S$ into clusters; rather, it is covering $S$ with clusters (each of cardinality at least $r$ and of MEB diameter at most $D$), and individual points of $S$ may be in many such clusters.  In fact, the {\em covering} version of the $r$-gather problem is solvable exactly in polynomial time in fixed dimension Euclidean spaces, since we can simply search for a value of $D$ (naively, or with binary search) such that all MEBs of diameter at most $D$, containing at least $r$ points of $S$, covers all of $S$.
%% Q: what about covering $S$ with the fewest sets each of DIAMETER at most $D$ (instead of MEB diameter), and also possibly containing at least $r$ points??  Can this problem be solved efficiently?  Note that there could be an exponential number of different such sets, but it should suffice to consider only those that are ``maximal'', and these may be polynomial (and computable using DP??)  Good question for CG Group!


%\pagebreak


%\medskip\noindent\textbf{Contributions.} 
\subsection{Our Results}
In this paper we investigate the $r$-gather problem in the Euclidean setting, in the dynamic/mobile setting, and in the decentralized setting. We obtain the following results. 
\bitem

\item We show new results on hardness of approximation for $r$-gather problem in the Euclidean setting. For minimizing the largest diameter of the clusters, we show that it is NP-hard to approximate better than a factor $2$ when $r\geq5$, and that it is NP-hard to approximate better than a factor $\sqrt{2+\sqrt{3}} \approx 1.931$ when $r=3$ or $4$.  Recall that the diameter of a set is the maximum distance between a pair of points in the set.

For minimizing the largest of the minimum enclosing balls (MEB) of the clusters, we show that it is NP-hard to approximate better than a factor ${\sqrt{35}+\sqrt{3} \over 4} \approx 1.912$ when $r \geq 4$, and that it is NP-hard to approximate better than a factor $\sqrt{13}/2 \approx 1.802$ when $r=3$.

\item In the mobile setting, we are given a set of trajectories. We show that if we  minimize the maximum distance of a pair at any time of the trajectory then the same $2$-approximate $r$-gather algorithm for points apply. When we allow $k$ regroupings for a given parameter $k$ and minimize the maximum diameter of any clusters, we show that one can use dynamic programming and obtain a $2$-approximtion. On the other hand, we show that in the worst case the number of cluster changes can be $\Omega(n^3)$, if we wish to maintain optimal $r$-gather solution at \emph{all} time.

\item We consider the decentralized setting and design a $4$-approximate algorithm. Each node makes local decisions and operations. The algorithm is based on a certain type of sweeping procedure. The sweep-clustering of a point depends only on local configurations, it is not influenced by outliers elsewhere in the network. This nice property ensures that the clustering is robust to noises/outliers. This algorithm can be naturally extended to the mobile setting and the solution adapts according to the mobile node mobility. 

\item Finally, we presented simulation results and comparisons of the various algorithms introduced here. 

\eitem

These results are reported in the same order in the next few sections. 