\documentclass[sigconf]{acmart}

\usepackage{booktabs}
\usepackage{mysymbols}
\usepackage{graphicx}
\usepackage{color}
%\usepackage{amsmath,amssymb,xspace,epsfig}
%\usepackage{cite}


\newcommand{\denselist}{\itemsep 0pt\parsep=0.8pt\partopsep 0pt}
\newcommand{\bitem}{\begin{itemize}\denselist}
\newcommand{\eitem}{\end{itemize}}
\newcommand{\benum}{\begin{enumerate}\denselist}
\newcommand{\eenum}{\end{enumerate}}



\newcommand{\myparagraph}[1]{\vspace*{1mm}\noindent {\bf #1}}
\newcommand{\rik}[1]{\vspace{2mm}\noindent {\bf \marginpar{***}\noindent Rik's comment:} #1\vspace{2mm}} 

\newcommand{\Frechet}{Fr\'echet }
\newcommand{\eps}{\varepsilon}
\newcommand{\dst}{\displaystyle}
%\newcommand{\abs}[1]{\left| #1 \right| }
\newtheorem{observation}[theorem]{Observation}
\newcommand{\ts}{\textsuperscript}


\acmConference[MOBIHOC '17]{ACM MobiHoc conference}{July 2017}{IIT Madras, Chennai, India.} 
\acmYear{2017}
%\copyrightyear{2016}





%compliles with pdflatex
\begin{document}
%\setcopyright{acmcopyright}

\title{Mobile $r$-gather: Distributed and Geographically Coherent Cardinality Clustering}

%Alternative discussed at meeting: \title{Distributed mobile $r$-gather: Geographically Coherent Cardinality Clustering

%\author{
%Rik Sarkar\authorrefmark{1} \hspace*{1.cm} Xianjin
%Zhu\authorrefmark{1} \hspace*{1.cm} Jie Gao\authorrefmark{1}
%\hspace*{1.cm} Leonidas J. Guibas\authorrefmark{2} \hspace*{1.cm}
%Joseph S. B. Mitchell\authorrefmark{3}\\ } \vspace*{.3cm}
%\small
%\authorrefmark{1}
%Department of Computer Science, Stony Brook University. \{rik,
%xianjin, jgao\}@cs.sunysb.edu \\ \authorrefmark{2} Department of
%Computer Science, Stanford University. guibas@cs.stanford.edu \\
%\authorrefmark{3} Department of Applied Mathematics and Statistics,
%Stony Brook University. jsbm@ams.sunysb.edu}


\author{Jiemin Zeng\ts{1}, Gaurish Telang\ts{2}, Matthew P. Johnson\ts{3},\\Rik Sarkar\ts{4}, Jie Gao\ts{5}, Esther Arkin\ts{6}, Joseph S. B. Mitchell\ts{6}}
\affiliation{\small \ts{1} Google Inc. jiemin.zeng@stonybrook.edu\\
\ts{2} Department of Applied mathematics and Statistics, Stony Brook University. gaurish.telang@stonybrook.edu\\
\ts{3} Department of Mathematics and Computer Science, Lehman College. mpjohnson@gmail.com\\
\ts{4} School of Informatics, University of Edinbubrgh. rsarkar@inf.ed.ac.uk\\
\ts{5} Department of Computer Science, Stony Brook University. jgao@cs.stonybrook.edu\\
\ts{6} Department of Applied mathematics and Statistics, Stony Brook University. \{estie,jsbm\}@ams.stonybrook.edu\\
}

\renewcommand{\shortauthors}{Zeng et. al.}
\renewcommand{\shorttitle}{Mobile r-gather: Distributed and Geographically Coherent Cardinality Clustering}

\begin{abstract}

Grouping mobile nodes into clusters can ease the management of a large number of devices and their information. Since applications and communication in mobile devices are highly location dependent, clustering by location is particularly useful in this context. In this paper, we consider the $r$-gather problem in which we must group nodes into clusters each having at least $r$ nodes (so that each cluster has a meaningful population), while minimizing the maximum diameter of the clusters (so that each cluster is geographically coherent).

We present several new results on the $r$-gather problem, including hardness of approximation in a variety of geometric settings and new distributed approximation algorithms for optimizing the maximum diameter of a cluster. In particular, our distributed algorithm for $r$-gather clustering allows computation to be pushed to the edges of the network. This method produces provably near-optimal results and can adapt to the dynamics of nodes in motion. The distributed approach naturally comes with the advantage of greater resilience. Additionally, we show that it achieves local optimality; i.e., from the point of view of any particular node, the solution is nearly as favorable as possible, irrespective of the global configuration. 

\end{abstract} 

\maketitle


%!TEX root = r-gather.tex

%\pagebreak

\section{Introduction}

We study the problem of clustering mobile nodes into meaningful, highly location-dependent clusters.  In order to quantify ``meaningful'' clusters, we establish a lower bound, $r$, on the size of clusters.  The resulting {\em $r$-gather} clustering problem is formally stated:
Given a set of $n$ points $P = \{p_1, p_2, \dots, p_n\}$ in Euclidean space and a value $r$, cluster the points into groups of at least $r$ points each such that the largest diameter of the clusters is minimized. We consider two popular notions of ``diameter'' of a cluster: the usual notion of {\em diameter} of a point set (the maximum distance between two points of the set), and the diameter of the minimum enclosing ball (MEB) of the set.

%\medskip\noindent\textbf{Motivations.} 
\subsection{Motivation}
One motivation of this version of clustering arises in location privacy in wireless networking. With the ubiquitous use of GPS receivers on mobile devices, it is now common practice that the locations of these mobile devices are recorded and collected. This raises privacy concerns as location information is sensitive and can be used to identify the user of the devices~\cite{blumberg2009}. This problem is more challenging when location is part of the input in location-based queries, for example, finding the coffee shop nearest to the user, querying traffic situations, and security-related applications such as reporting suspicious behaviors. In these settings, the user submits a query to location-based services (LBS) through a mobile device. An adversary that compromises the LBS server can infer private information about the user. In this setting, protection of privacy is characterized into two different yet related types: \emph{query privacy}, e.g., whether an adversary can identify the user who issued the query (i.e., associate user IDs with queries), and \emph{location privacy}, e.g., how much an adversary can learn regarding location of a user. %Most privacy protection algorithm use location perturbation and obfuscation methods. 
A nice survey on this topic can be found in~\cite{Chow:2011:TPL:2031331.2031335,6155874}.

For query privacy, the most common approach is to use the $k$-anonymity measure~\cite{Sweeney02}, which says that the locations are grouped into clusters, each of at least $k$ points. 
In previous work~\cite{Gruteser:2003:AUL:1066116.1189037,Mokbel:2006:NCQ:1182635.1164193} ``cloaking boxes'' were used to group spatio-temporal user queries into a box with at least $k$ queries and then submit the box (instead of the queries themselves) to the LBS server. In this way, the query sender is indistinguishable from the $k-1$ other query users in the same box. It is ideal to group queries from similar locations into the same box.
%One common practice in the treatment of location data is to adopt the $k$-anonymity criterion~\cite{Sweeney02}, which says that the locations are grouped into clusters, each of at least $k$ points. The cluster is used to replace individual locations, so that any one user is not differentiated from the $k-1$ others. 
Towards this goal, one objective to minimize the diameter of the clusters, since this yields location data with the best possible accuracy, while not intruding on user privacy.  This is precisely the $r$-gather problem~\cite{Aggarwal06achievinganonymity}.

%\pagebreak

%Protecting location privacy was brought to the attention of the research community a few years ago~\cite{blumberg2009}. A number of work has been conducted to protect the privacy of a \emph{single} location, when the user submits a query to location-based services (LBS) on mobile phones. An adversary that compromises the LBS server can infer private information about the user. In this setting, protection of privacy is characterized into two different yet related types: \emph{query privacy}, e.g., whether an adversary can identify the user who issued the query, and \emph{location privacy}, e.g., whether a user can be accurately located. Most privacy protection algorithm use location perturbation and obfuscation methods. A nice survey can be found in~\cite{Chow:2011:TPL:2031331.2031335,6155874}.

%For location privacy, one common metric used is location entropy, which characterizes the uncertainty of the location information an adversary can extract from LBS queries. Most methods use perturbations of the true locations to confuse the adversary (for example, see~\cite{1607571,4497445,Hong:2004:APU:990064.990087}). Notice that the reported location cannot be too far away from the true location as otherwise the location based query will become useless. There is a tradeoff of location privacy versus query utility.

Apart from protecting the privacy of a single snapshot, which is formulated by the static $r$-gather problem, it is natural to consider the dynamic setting for mobile users. 
%Protecting the privacy of the trajectory data is not investigated as much. 
%This is partly because the unique geometric structures in the trajectory data that call  for novel geometric algorithms. Previous work is along the following directions. %A small number of prior work was available.
%In~\cite{1506394} and~\cite{You:2007:PMT:1548880.1548998}, dummies with movement patterns similar to those of real users are generated and sent to the LBS to confuse the adversary. But two problems remain: the adversary may differentiate real users from dummies after long time tracking, and the added noises alter the properties of the data set and consequently decrease the data utility. %In~\cite{You:2007:PMT:1548880.1548998}, the original trajectories are first analyzed in three metrics about the `degree of privacy protection': snapshot disclosure, trajectory disclosure and distance deviation. Then dummy trajectories are added until the final set satisfies the requirement. One issue to be aware of for the inclusion of dummies is how much the added noises alter the properties of the data set and consequently decrease the data utility.
When mobile phones continuously issue location-based services, temporal spatial cloaking boxes are created~\cite{Chow:2007:EPC:1784462.1784477,Xu:2007:LAC:1341012.1341062,4509698}. The challenge is that the cloaking box may become huge after a long time period, leading to high computational cost and low query accuracy. Clearly there is a tradeoff between the quality of the cluster size and the number of re-clusterings we want to afford. Formulating this problem using the $r$-gather framework, we may ask the follow question: Given a parameter $k$ specifying the maximum number of times we can recluster over the time period of interest, when and how should re-clustering be done so that the maximum cluster diameter is minimized, while satisfying the $r$-gather constraint that each cluster have at least $r$ elements?

Further, the $r$-gather problem has been investigated in centralized settings, when all information is gathered at a single location. In mobile applications, it is desirable to  develop distributed algorithms, with local operations and computations distributed among the nodes. 

Besides the natural connection to location privacy issues, the $r$-gather problem is a natural and useful variaent of mobile clustering in general. Many mobile applications rely on grouping the mobile nodes into clusters for management purposes. These clusters need to be geographically coherent and for a cluster to be meaningful, it is natural to require each cluster to have a minimum number of members. Clustering mobile nodes has been studied a lot in the past, for example in~\cite{593002,lin97adaptive,chen99clustering,basu01mobility,Basagni99distributed,gghzz-dmc-03}. But none of the previous work strictly enforce a lower bound on the cluster cardinality. 

%A different idea~\cite{1186725} introduces `mixed-zones', a geometric region such that users entering the zone change to a pseudonym and do not send LBS queries. If the mixed zone has $k$ users, the probability of each user leaving the zone along each exit is the same and each user spends a random duration of time inside the zone, the set of users are $k$-anonymized. Such mixed zones are created both as geographical boxes or segments along road network~\cite{LCA-CONF-2007-016,5767898}. This scheme requires hardware support to perform pseudonym change. It also imposes limitations to user queries within the zone.

%In the case of protecting privacy during publication of trajectories to a third party, two ideas have been mainly used: clustering based~\cite{4497446,Gao:2014:BTP:2567003.2567226} and generalization based~\cite{Nergiz:2009:TTA:1556406.1556410,Monreale:2010:MDA:1824401.1824403}. A set of trajectories have two distinct structures -- the coordinates of data points in each temporal snapshot, and how these data points are linked to form trajectory paths. The clustering based approach alters the former while the generalization based~\cite{Nergiz:2009:TTA:1556406.1556410,Monreale:2010:MDA:1824401.1824403} algorithm alters the later. In clustering based approach, the trajectory data is clustered into groups of $k$ co-located trajectories within the same time period to form a $k$-anonymized aggregated trajectory. The limitation of this approach is the strong requirement that such clusters can be found in the data set. In addition, many applications require atomic trajectories and are not ready to take aggregated trajectories. In generalization based approach, each trajectory is first transformed into a sequence of $k$-anonymized regions. Then for each trajectory, the algorithm uniformly select $k$ atomic points in each anonymized region and link a unique atomic point from each region to reconstruct $k$ trajectories. Performance of this approach can be evaluated in two aspects -- how big the anonymized regions are and how much the altered data change the utility of this data set. The scheme by \cite{Chen:2013:PTD:2442161.2442241} can be considered as a mixture of clustering and generalization. Besides, a few other papers make separate assumptions on the power of the adversary and consider different metrics~\cite{Terrovitis:2008:PPP:1397755.1397842,Yarovoy:2009:AMO:1516360.1516370}.

%A number of work has been done using differential privacy~\cite{Dwork:2011:FFP:1866739.1866758,Blum:2008:LTA:1374376.1374464,Dwork:2009:CDP:1536414.1536467,Xiao:2011:DPV:2006855.2007020,Xiao:2010:DPD:1889159.1889173,Jiang:2013:PTD:2484838.2484846,Hu:2010:PLD:1806907.1806910}. Most of the work focus on relational databases. Trajectory data has strong sequentiality. The methods for relational databases typically face scalability issues in the context of trajectory data~\cite{Mohammed:2011:DPD:2020408.2020487,journals/pvldb/ChenMFDX11}.

%!TEX root = r-gather.tex

\subsection{Related Work}

The $r$-gather problem has been studied for instances in general metric spaces.  Aggarwal et al.~\cite{Aggarwal06achievinganonymity} give a $2$-approximation algorithm and show that, for $r>6$, it is NP-hard to approximate with an approximation ratio better than $2$.  The approximation algorithm first guesses the optimal diameter, then greedily selects clusters with twice the diameter; finally, a flow algorithm is used to assign at least $r$ points to each cluster.  This procedure is repeated until a good choice of diamter is found.  Note that this solution only selects input points as cluster centers.

Armon \cite{armon2011min} extended the result of Aggarwal et al. proving that, for $r>2$, it is NP-hard to approximate with a ratio better than $2$ for the case of general metric spaces.  Armon also considers a generalization of the $r$-gather clustering problem, called the {\em $r$-gathering problem}, which also considers a given set of potential cluster centers (potential ``facility locations''), each having a fixed set-up cost that is included in the objective function. Armon provides a $3$-approximation for the min-max $r$-gathering problem and proves that it is NP-hard to obtain a better approximation factor.  Additional results include various approximation algorithms for the min-max $r$-gathering problem with a {\em proximity requirement} that each point be assigned to its nearest cluster center.

For the case $r = 2$, both \cite{anshelevich2011terminal} and \cite{shalita2010efficient} provide polynomial-time exact algorithms.  Shalita and Zwick's \cite{shalita2010efficient} algorithm runs in $O(mn)$ time, for a graph with $n$ nodes and $m$ edges.

All of these prior algorithms were for the centralized setting, in general metrics; we are not aware of results in the distribtued and dynamic settings, or of results that exploit geometric structure. 

%% Joe added this discussion after submission to INFOCOM:
A related problem is that of the discrete disk cover: Given a set $S$ of points in a metric space, and a collection ${\cal B}$ of disks whose union covers $S$, find a smallest cardinality subset of ${\cal B}$ that covers $S$.  As a set cover instance, it is well known that the greedy algorithm yields a logarithmic factor approximation.  %%% xxx add a citation?  CLRS?
In geometric instances, e.g., in low-dimensional Euclidean space, it is known that local search yields a PTAS.  %% xxx cite Mustafa and Ray, and newer papers??
Since the number of combinatorially distinct disks is polynomial, in any fixed dimension Euclidean space, one could specify ${\cal B}$ to be the set of all (minimal) disks that contain at least $r$ points of $S$ and that have diameter at most some parameter $D$.  Minimizing the number of such disks in a cover of $S$, then, has a PTAS, from known results on discrete disk cover.  However, this problem is not the same as the $r$-gather cluster problem, since it is not partitioning the points of $S$ into clusters; rather, it is covering $S$ with clusters (each of cardinality at least $r$ and of MEB diameter at most $D$), and individual points of $S$ may be in many such clusters.  In fact, the {\em covering} version of the $r$-gather problem is solvable exactly in polynomial time in fixed dimension Euclidean spaces, since we can simply search for a value of $D$ (naively, or with binary search) such that all MEBs of diameter at most $D$, containing at least $r$ points of $S$, covers all of $S$.
%% Q: what about covering $S$ with the fewest sets each of DIAMETER at most $D$ (instead of MEB diameter), and also possibly containing at least $r$ points??  Can this problem be solved efficiently?  Note that there could be an exponential number of different such sets, but it should suffice to consider only those that are ``maximal'', and these may be polynomial (and computable using DP??)  Good question for CG Group!


%\pagebreak


%\medskip\noindent\textbf{Contributions.} 
\subsection{Our Results}
In this paper we investigate the $r$-gather problem in the Euclidean setting, in the dynamic/mobile setting, and in the decentralized setting. We obtain the following results. 
\bitem

\item We show new results on hardness of approximation for $r$-gather problem in the Euclidean setting. For minimizing the largest diameter of the clusters, we show that it is NP-hard to approximate better than a factor $2$ when $r\geq5$, and that it is NP-hard to approximate better than a factor $\sqrt{2+\sqrt{3}} \approx 1.931$ when $r=3$ or $4$.  Recall that the diameter of a set is the maximum distance between a pair of points in the set.

For minimizing the largest of the minimum enclosing balls (MEB) of the clusters, we show that it is NP-hard to approximate better than a factor ${\sqrt{35}+\sqrt{3} \over 4} \approx 1.912$ when $r \geq 4$, and that it is NP-hard to approximate better than a factor $\sqrt{13}/2 \approx 1.802$ when $r=3$.

\item In the mobile setting, we are given a set of trajectories. We show that if we  minimize the maximum distance of a pair at any time of the trajectory then the same $2$-approximate $r$-gather algorithm for points apply. When we allow $k$ regroupings for a given parameter $k$ and minimize the maximum diameter of any clusters, we show that one can use dynamic programming and obtain a $2$-approximtion. On the other hand, we show that in the worst case the number of cluster changes can be $\Omega(n^3)$, if we wish to maintain optimal $r$-gather solution at \emph{all} time.

\item We consider the decentralized setting and design a $4$-approximate algorithm. Each node makes local decisions and operations. The algorithm is based on a certain type of sweeping procedure. The sweep-clustering of a point depends only on local configurations, it is not influenced by outliers elsewhere in the network. This nice property ensures that the clustering is robust to noises/outliers. This algorithm can be naturally extended to the mobile setting and the solution adapts according to the mobile node mobility. 

\item Finally, we presented simulation results and comparisons of the various algorithms introduced here. 

\eitem

These results are reported in the same order in the next few sections. 
%%!TEX root = r-gather.tex

\subsection{Related Work}

The $r$-gather problem has been studied for instances in general metric spaces.  Aggarwal et al.~\cite{Aggarwal06achievinganonymity} give a $2$-approximation algorithm and show that, for $r>6$, it is NP-hard to approximate with an approximation ratio better than $2$.  The approximation algorithm first guesses the optimal diameter, then greedily selects clusters with twice the diameter; finally, a flow algorithm is used to assign at least $r$ points to each cluster.  This procedure is repeated until a good choice of diamter is found.  Note that this solution only selects input points as cluster centers.

Armon \cite{armon2011min} extended the result of Aggarwal et al. proving that, for $r>2$, it is NP-hard to approximate with a ratio better than $2$ for the case of general metric spaces.  Armon also considers a generalization of the $r$-gather clustering problem, called the {\em $r$-gathering problem}, which also considers a given set of potential cluster centers (potential ``facility locations''), each having a fixed set-up cost that is included in the objective function. Armon provides a $3$-approximation for the min-max $r$-gathering problem and proves that it is NP-hard to obtain a better approximation factor.  Additional results include various approximation algorithms for the min-max $r$-gathering problem with a {\em proximity requirement} that each point be assigned to its nearest cluster center.

For the case $r = 2$, both \cite{anshelevich2011terminal} and \cite{shalita2010efficient} provide polynomial-time exact algorithms.  Shalita and Zwick's \cite{shalita2010efficient} algorithm runs in $O(mn)$ time, for a graph with $n$ nodes and $m$ edges.

All of these prior algorithms were for the centralized setting, in general metrics; we are not aware of results in the distribtued and dynamic settings, or of results that exploit geometric structure. 

%% Joe added this discussion after submission to INFOCOM:
A related problem is that of the discrete disk cover: Given a set $S$ of points in a metric space, and a collection ${\cal B}$ of disks whose union covers $S$, find a smallest cardinality subset of ${\cal B}$ that covers $S$.  As a set cover instance, it is well known that the greedy algorithm yields a logarithmic factor approximation.  %%% xxx add a citation?  CLRS?
In geometric instances, e.g., in low-dimensional Euclidean space, it is known that local search yields a PTAS.  %% xxx cite Mustafa and Ray, and newer papers??
Since the number of combinatorially distinct disks is polynomial, in any fixed dimension Euclidean space, one could specify ${\cal B}$ to be the set of all (minimal) disks that contain at least $r$ points of $S$ and that have diameter at most some parameter $D$.  Minimizing the number of such disks in a cover of $S$, then, has a PTAS, from known results on discrete disk cover.  However, this problem is not the same as the $r$-gather cluster problem, since it is not partitioning the points of $S$ into clusters; rather, it is covering $S$ with clusters (each of cardinality at least $r$ and of MEB diameter at most $D$), and individual points of $S$ may be in many such clusters.  In fact, the {\em covering} version of the $r$-gather problem is solvable exactly in polynomial time in fixed dimension Euclidean spaces, since we can simply search for a value of $D$ (naively, or with binary search) such that all MEBs of diameter at most $D$, containing at least $r$ points of $S$, covers all of $S$.
%% Q: what about covering $S$ with the fewest sets each of DIAMETER at most $D$ (instead of MEB diameter), and also possibly containing at least $r$ points??  Can this problem be solved efficiently?  Note that there could be an exponential number of different such sets, but it should suffice to consider only those that are ``maximal'', and these may be polynomial (and computable using DP??)  Good question for CG Group!


%\pagebreak

%!TEX root = r-gather.tex

\section{A Distributed Algorithm for \lowercase{$r$-gather}}


In this section, we consider the $r$-gather as a distributed computation problem. This approach is particularly relevant in the context of locations, where data is naturally spread over a spatial region, and we can use local computations at access points and local devices for anonymization and cloaking. This approach also provides better security and privacy, since it is harder for an attacker to compromise many devices spread over a large region.

\subsection{Distributed Computation and Location Management}

We consider the problem from an {\em edge computation} perspective, where computation is pushed away from central servers and toward the edges of the network. Computations may be carried out in mobile phones themselves, or in other local facilities, such as access points, cellular base stations or other local servers. In such setups, a mobile device may not need to perform its own computations, which may instead be performed by servers in charge of each locality. 

We assume that each mobile device is capable of finding its approximate location, either from GPS or from the presence of nearby transmitters (e.g., cell towers). We also assume that the devices report their location changes to a distributed location management system such as~\cite{abraham04LLS}. Such location management systems can be modified easily to respond to range queries, such as how many nodes are present in a given area~\cite{Sarkar:2010:forms}. In the following, we assume that every node can query the location server to find nodes within any particular distance from it, and thus derive its $r$ nearest neighbors. We follow the general distributed computation terminology of a node performing computations, but,  in general, location servers may be carrying out the computations on behalf of the nodes. We give more details of location management and neighborhood queries in Subsection~\ref{subsec:dynamic}


\subsection{Maximal Independent Neighborhoods}

We assume there is a set of $n$ mobile nodes $1,2,\dots , n$, and use $p_{i}$ to denote the location of node $i$. The set $P$ is the set of locations ($p_{i}$s) of the nodes. 
For any point $p_i$, we let $p_i^{(r)}$ denote its $r^{th}$ nearest neighbor in $P$, and we let $d_{r}(p_{i}) = \abs{p_{i} - p_{i}^{(r)}}$ denote the corresponding distance. We let $N_{r}(p_{i})$ denote the set of $r$ nodes nearest to point $p_i$, and let $N(P)=\{N(p_{i}): p_{i}\in P\}$ denote the set of such {\em $r$-neighborhoods}. If $N_{r}(p_{i}) \cap N_{r}(p_{j})=\emptyset$, we say that the $r$-neighborhoods of $p_{i}$ and $p_{j}$ are {\em independent}.

%Our general strategy will be to create disks in the plane, with each disk containing at least $r$ nodes. Thus, centered at a point $p_{i}$, we consider disks of radius $d_{r}(p_{i})$, which we denote by $D_{r}(p_{i})$. We call two disks $D$ and $D^{\prime}$ independent if the nodes in them do not intersect, that is, $D\cap P \cap D^{\prime} = \emptyset$. 

We let $\G$ denote the set of clusters, at any stage of our algorithm; $\G$ is initialized to $\emptyset$. 
% NOT USED at the moment: We let $G(p_{i}) \in \G$ denote the cluster that contains node $p_{i}$. 
We let $c_G$ denote the {\em center} of cluster $G\in \G$; the center $c_G$ may be either a node (in $P$) or another location.

The basic algorithm executes the following steps to construct the set $\G$ of clusters:
\begin{enumerate}
\item[M1] At each point $p_{i}\in P$, compute $p_{i}^{(r)}$, $d_{r}(p_{i})$ and $N_{r}(p_{i})$.
\item[M2] Find a maximal independent subset of neighborhoods from the set $N_{r}(P)$, add each as a cluster in $\G$, and {\em mark} the nodes (as ``clustered'') in these sets.
\item[M3]  For any unmarked node $p_{i}\in P$, assign $p_{i}$ to the cluster $G\in\G$ whose center, $c_G$, is closest to $p_{i}$.
%% with the nearest center, that is $\dst\argmin_{G^{j}}{\abs{p_{i}-p_{j}}}$ and mark $i$ as clustered. 
\end{enumerate}

The nodes that belong to $r$-neighborhoods of cluster centers and added to clusters in step M2 are called {\em canonical} members of the cluster, while nodes that are added in step M3, are called the {\em outer} members. 


%In this section, we describe a 4-approximation algorithm for $r$-gather that is less centralized than the 2-approximation in \cite{Aggarwal06achievinganonymity}.  We begin with an algorithm that is not explicitly decentralized and then later detail how to do so.

%Let the $r$-neighborhood of a point $p_i$ or $N_r(p_i)$ denote the set containing $p_i$ and the closest $r-1$ points to $p_i$.  Let $N$ be the set of the $r$-neighborhoods of all points in $P$.  For each $r$-neighborhood, we define a distance $R_i^r = \max_{p_j \in N_r(p_i)}||p_i - p_j||$ and we define a distance $R^r = \max_{1 \leq i \leq n}R_i^r$ among all $r$-neighborhoods.  We first find a maximal independent set $S$ of $r$-neighborhoods.  For an $r$-neighborhood $N_r(p_i)$, we name $p_i$ the center of the cluster and all other points in $N_r(p_i)$ are named the cannonical set.  Each point $p_i$ that is not in a set in $S$ must have at least one point in it's $r$-neighborhood that is in a set in $S$ (otherwise $S$ is not maximal).  We assign $p_i$ to the set of one of these points.  Such a point is named an outer member of its set.  We claim that the resulting clustering $S'$ is a 4-approximation $r$-gather clustering.  


Next, we argue that this simple algorithm approximates an optimal clustering. Let $d_{r}^{\max} = \max_{i}{d_{r}(p_{i})}$ be the largest distance from a node to its $r^{th}$ nearest neighbor. Let $D_{OPT}$ be the diameter of the largest cluster in an optimal clustering.  We then observe

\begin{observation}
%%For any set of nodes in the plane, 
$D_{OPT}\geq d_{r}^{\max}$.
\end{observation}
\begin{proof}
Suppose $p_{i}\in P$ is a point achieving $d_{r}^{\max}$: $d_{r}(p_{i}) = d_{r}^{\max}$. Since any disk of radius less than $d_{r}^{max}$ centered at $p_{i}$ does not contain $r$ nodes, a disk that contains $p_{i}$ as well as (at least) $r-1$ other nodes must have radius at least $d_{r}^{max}/2$. Thus the diameter $D_{OPT}$ must be at least $d_{r}^{\max}$.
\end{proof}

\begin{lemma}
For any cluster $G\in \G$ and any node $p_{i}\in G$, $|p_{i} - c_G| \leq d_r(p_{i})+d_r^{\max}$.   % $ d_r(p_{i})+d_r(c_G)$.
%%  distance from $x$ to $j$ is bounded by $\abs{p_{x} - p_{j}}\leq d_{r}(p_{j}) + d_{r}(p_{x})$. 
\label{lem:local-lemma}
\end{lemma}
\begin{proof}
Consider a cluster $G\in\G$, centered at $c_G$, and $p_{i}\in G$. 

If $p_{i}$ was assigned to cluster $G$ in step M2, then we know that $p_{i}\in N_r(c_G)$, implying that $|p_{i} - c_G| \leq d_r(c_G) \leq d_r(p_{i})+ d_r(c_G)$.

% All nodes added to $G$ in step M2 of the algorithm are within a distance $d_{r}(c_G)$ of $c_G$; thus, all nodes in $G$ are within distance $2d_{r}(c_G)$ of each other. 

If $p_{i}$ was not assigned to cluster $G$ in step M2, but was instead assigned to $G$ in step M3, then we know, by maximality of the independent set, that the $r$-neighborhood $N_r(p_i)$ intersects some other $r$-neighborhood, say $N_r(p_j)$, that was a cluster in the maximal independent set in step M2.  (It may or may not be the case that $G=N_r(p_j)$.)  
Thus, there is a node $p_y \in N_r(p_i)\cap N_r(p_j)$, implying that $|p_i - p_y|\leq d_r(p_i)$ and that $|p_y - p_j|\leq d_r(p_j)$.  The triangle inequality implies then that $|p_i-p_j|\leq |p_i-p_y|+|p_y-p_j|\leq d_r(p_i)+d_r(p_j)\leq d_r(p_i)+d_r^{\max}$.  
Since $p_i$ is closer to $c_G$ than to the alternative center $p_j$, we get the claimed inequality, 
$|p_i-c_G|\leq |p_i-p_j|\leq d_r(p_i)+d_r^{\max}$. 
% Since the disk of radius $d_r(p_i)$ centered at $p_i$ intersects the disk of radius $d_r(p_j)$ centered at $p_j$, and $p_i$ is closer (or at the same distance) to $c_G$ than to $p_j$, we know that the disk of radius $d_r(p_i)$ centered at $p_i$ also intersects the disk of radius $d_r(c_G)$ centered at $c_G$. Thus, there is a point $q$ in the intersection of these two disks (the disk of radius $d_r(p_i)$ centered at $p_i$ and the disk of radius $d_r(c_G)$ centered at $c_G$), implying that $|p_i - q|\leq d_r(p_i)$ and that $|q - c_G|\leq d_r(c_G)$.  The triangle inequality implies then that $|p_i-c_G|\leq |p_i-q|+|q-c_G|\leq d_r(p_i)+d_r(c_G)$.  
%Now consider a node $p_j\in P$ that is not assigned to a cluster in step M2; $p_j$ is assigned to a cluster in step M3. That $x$ is not clustered implies that one or more nodes in $N_{r}(p_{x})$ belongs to some other neighborhood $N_{r}(p_{j)}$ for a cluster $G^{j}$. Let $y\in N_{r}(p_{x})\cap N_{r}(p_{j})$ be such a node. Then by definition, $\abs{p_{x}-p_{y}}\leq d_{r}(p_{x})$ and $\abs{p_{y} - p_{j}}\leq d_{r}(p_{j})$. Thus, by triangle inequality, $\abs{p_{x} - p_{j}}\leq d_{r}(p_{j}) + d_{r}(x)$.
\end{proof}

From the above lemma it follows that the algorithm produces a $4$-approximation of the diameter:

\begin{corollary}
The diameter of any $G\in \G$ is at most $4D_{OPT}$.
\end{corollary}
\begin{proof}
Consider any $p_i,p_j\in G$.  By Lemma~\ref{lem:local-lemma}, 
$|p_i-c_G|\leq d_r(p_i)+d_r^{\max}\leq 2d_r^{\max}$ and
$|p_j-c_G|\leq d_r(p_j)+d_r^{\max}\leq 2d_r^{\max}$.
Thus, by the triangle inequality, $|p_i-p_j|\leq 2d_r^{\max} + 2d_r^{\max} = 4d_r^{\max} \leq 4D_{OPT}$. 
%Since any $d_{r}(\cdot)\leq d_{r}^{\max}$, it follows using triangle inequality that: \[ \abs{p_{x} - p_{z}}\leq \abs{p_{x}-p_{j}} + \abs{p_{y}-p_{j}}\leq  4D_{OPT}\]
\end{proof}

%\myparagraph{Randomized algorithm.} 
The maximal independent subsets in step M2 can be computed rapidly, in time $O(\log n)$, using the randomized parallel algorithm of Alon et al.~\cite{alon1986fast}, applied to compute a maximal independent set in the intersection graph of the neighborhoods $N(P)$ (i.e., in the graph whose nodes are the $r$-neighborhoods $N(p_{i})$ and whose edges link two $r$-neighborhoods that have a nonempty intersection).

%% as follows. We construct a graph $H$ on the set of nodes such that any edge $(i,j)$ exists iff $N_{r}(p_{i})\cap N_{r}(p_{j})\neq \emptyset$. The maximal independent subset of neighborhoods in step M2 is then equivalent to computation of a maximal independent set on this graph $H$. We can then use a randomized computation such as~\cite{alon1986fast} to compute the independent neighborhoods in $O(\log n)$ time.

The algorithm just described guarantees a $4$-approximation overall; however, from the point of view of a particular node this may not be satisfactory. The bound on the diameter for all clusters is dominated by the worst case -- the clusters in the sparsest neighborhood. A node in a densely populated region can justifiably expect to be assigned to a cluster center close to itself, which is not guaranteed by the algorithm above. We thus describe next another algorithm that guarantees geographic coherence of clusters, meaning that the distance of a node $p_{i}$ to its cluster center is bounded by a factor of its distance, $d_{r}(p_{i})$, to its $r^{th}$ nearest neighbor. 

\subsection{Distributed Sweep Algorithm with Coherence Guarantee}

In this strategy we create clusters in the dense regions first, and then move to sparser regions. 

\myparagraph{Finding maximal independent sets.} At each node $p_i$, we consider the function $d_{r}(p_{i})$, and we assume, without loss of generality, that the function values $d_r(p_i)$ are distinct (ties can be broken according to node id numbers).
% at the node and at its $r$ nearest neighbors. We assume that $d_{r}$ has a unique value at every node, and break ties by node id. 
Each node $p_i$ maintains two variables:
\begin{itemize}
\item Its cluster center pointer, intialized to NULL. When node $p_i$ is assigned a cluster, its cluster center pointer is assigned. 
\item A decision state, {\em decided/undecided}, to indicate whether $p_i$ is still in contention for becoming a cluster center.
\end{itemize}
Each node $p_i$ is initially in contention to become cluster center; we prefer nodes $p_i$ with smaller values of $d_{r}(p_i)$. The algorithm operates in rounds, as follows. In each round:

\begin{enumerate}
\item Every undecided and unclustered node $p_i$ requests permission from nodes in $N_{r}(p_{i})$ to become cluster center.
\item If all nodes in $N_{r}(p_{i})$ grant permission, then $p_i$ becomes a cluster center, and all nodes in $N_{r}(p_{i})$ are marked as {\em clustered} and {\em decided}. Additionally, they all set their cluster center pointer to $p_i$. 
\item If one or more nodes in $N_{r}(p_{i})$ {\em deny} permission for $p_i$ to become cluster center, then $p_i$ marks itself as {\em decided}, implying that it will not try to become cluster center any more.
\end{enumerate}

Any node $p_j$ that receives a permission request from $p_i$ responds as follows:
\begin{enumerate}
\item  If $p_j$ is unclustered {\em and} all {\em undecided} nodes $p_{j'}\in N_{r}(p_{j})$ have values of $d_{r}(p_{j'})$ greater than $d_r(p_i)$, then node $p_j$ gives permission to $p_i$; else, 
\item if $p_j$ is already clustered, then $p_j$ {\em denies} permission to $p_i$; else, 
\item if $p_j$ is not clustered, then $p_j$ {\em defers} permission to $p_i$.
\end{enumerate}

This approach essentially performs a sweep, starting from the densest regions of the network, working towards the less dense regions. The nodes with their $r$ nearest neighbors the closest have a chance to become cluster centers, while other nodes have to wait until these clusters have been formed. Once nodes in dense regions have been clustered into tight clusters, or have decided that they cannot form an independent cluster, nodes in neighboring sparser regions get the chance to become cluster centers. 

Any node left unclustered after the above process is assigned to the cluster of the nearest center, as in step M3. 

\begin{theorem}
If node $p_i$ belongs to cluster $G$ with center $c_G$, then $|p_{i} - c_G|\leq 2 d_{r}(p_{i})$. 
\end{theorem}
\begin{proof}
If $p_i=c_G$ then the claim is trivially true. If not, then there exists a node $p_y \in N_{r}(p_i)\cap N_{r}(p_j)$ for some cluster center $p_j$. Without loss of generality, suppose $p_j$ is the first such center, that is, the one with smallest $d_{r}(p_j)$. Then, $d_{r}(p_j)\leq d_{r}(p_i)$, since otherwise $p_j$ could not have been a center before $p_i$. Thus $\abs{p_{i}-p_{j}}\leq |p_i-p_y|+|p_y-p_j|\leq d_r(p_i)+d_r(p_j)\leq 2 d_{r}(p_i)$.
% using Lemma~\ref{lem:local-lemma}. 
If $p_j=c_G$, then this concludes the proof. If $p_j\neq c_G$, then since in step M3 each node is assigned to the nearest center, we have $\abs{p_{i}-c_G}\leq \abs{p_{i}-p_{j}}\leq 2 d_{r}(p_i)$.
\end{proof}

This proof implies that the center assigned to any node is at most twice the distance to its $r^{th}$ nearest neighbor, irrespective of locations of rest of the point set. 



\section{The $r$-Gather Problem in the Mobile Setting}

We continue our investigation of the $r$-gather problem by now considering the dynamic situation in which the points are in motion. There are various models to consider in the mobile setting; in this section, we consider several versions.  In each case, we assume that the trajectories of the mobile agents are piecewise-linear.

\medskip\noindent\textbf{Clustering Trajectories.} In the simplest formulation of $r$-gather in a mobile setting, we are given a set of trajectories over a time period $T$ and we want to cluster the trajectories such that each cluster has at least $r$ trajectories and the largest diameter of each cluster over the entire time period is minimized.  Here we designate the diameter of a cluster at a single point in time to be the distance between the furthest pair of points.  Points are assigned to a single cluster for the entire length of $T$ and do not switch clusters.  We claim that the $2$-approximation strategy in~~\cite{Aggarwal06achievinganonymity} for static $r$-gather in any metric space can also be applied to this problem.  
%We also claim that 2-inapproximation lower bound from Aggarwal et. al. applies as well \cite{Aggarwal06achievinganonymity}.  
We use the distance metric defined by Lemma~\ref{lem:distancemetric}.  

\begin{lemma}\label{lem:distancemetric}
The distance function $d_t(p,q)$ between two trajectories $p$ and $q$ over a time period $T$ is defined as the distance between $p$ and $q$ at time $t \in T$.  Then $d(p,q) = \max_{t \in T}d_t(p,q)$.  The function $d(p,q)$ is a metric.
\end{lemma}
\begin{proof}
The function by definition is symmetric, follows the identity condition, and is always non-negative.  To show that the metric follows the triangle equality, we first assume that there is a pair of trajectories $x$ and $z$ where $d(x,z) > d(x,y) + d(y,z)$ for some $y$.  There is some time $t \in T$, where $d_t(x,z) = d(x,z)$.  By the triangle inequality,
$$d_t(x,z) \leq d_t(x,y) + d_t(y,z).$$
In addition, clearly $d_t(x, y) \leq d(x, y)$, and $d_t(y, z)\leq d(y, z)$. 
This contradicts our assumption and concludes our proof.
\end{proof}




%\begin{theorem}
%For the definition of dynamic $r$-gather where all trajectories are clustered once, it is NP-hard to approximate better than a factor of 2 for $r > 6$.
%\end{theorem}

%\begin{proof}
%We construct a reduction from the problem 3-SAT where each variable is restricted to 3 clauses.  Our proof is similar to the lower bound proof in \cite{Aggarwal06achievinganonymity}.  Given a boolean formula in 3-CNF form with $m$ clauses $C_j$, $1 \leq j \leq m$, composed of $n$ variables $x_i$, $1 \leq i \leq n$, where $i$ and $j$ are integers, we construct a set of trajectories over a time period $T$.  For every variable $x_i$ and its complement $\bar{x_i}$, we have two points $v_i^T$ and $v_i^F$ and an additional $(r-2)$ points $u_i^k$ for integer $k$, $1 \leq k \leq r-2$.  We will construct trajectories where $d(v_i^T, v_i^F) = d(v_i^T, u_i^k) = d(v_i^F, u_i^k) = 1$, for $1 \leq i \leq n$ and $1 \leq k \leq r-2$ .  In addition, for every cluster $C_j$, we construct another point $w_j$.  We construct trajectories such that the distance between $w_j$ and the points that represent the literals in the clause $C_j$ is 1.  All other distances are 2.

%Constructing the trajectories that obey these distance constraints is simple.  Place all points at the origin on a number line.  For every point $p$, we have one time step that establishes the distance between it and all other points.  In that time step, $p$ moves to 2 on the number line.  At the same time, all other points that must have a distance of 1 to $p$ move to 1 on the number line.  This procedure is repeated for every point.

%We claim, that if there is an $r$-gather clustering of maximum radius 1, then the 3-SAT formula can be satisfied.  Here, for every variable, we have two possibilities for the center of a cluster of radius 1, $v_i^T$ and $v_i^F$.  There are not enough points close to $v_i^T$ and $v_i^F$ to have both as the center of two clusters.  Any cluster with $u_i^k$ as the center will have fewer than $r$ points within a distance of 1 and therefore cannot be a cluster center with radius 1.  Finally, every point $w_j$ has only three other points within a distance of 1.  Therefore, in any $r$-gather clustering with maximum radius 1, there is one cluster for every variable with a center of either $v_i^T$ or $v_i^F$.  This clustering must be done in such a way that for every point $w_j$, at least one of the points that represents a literal in $C_j$ must be chosen as a cluster center.  Thus if an optimal $r$-gather clustering of our trajectories can be found, then we can determine if the corresponding 3-SAT formula is satisfiable.
%\end{proof}

\medskip\noindent\textbf{Clustering Trajectories with Re-grouping.} Now, we consider the more general setting in which the clusters are allowed to change, via regroupings, during the time period $T$.  We let $k$ be a parameter that specifies the maximum number of regroupings allowed during $T$.  Each regrouping allows all clusters to be modified or changed completely.  The lower bounds for the earlier version of $r$-gather apply here as well, for the same reasons.  We claim that with the assumption that the trajectories are piecewise-linear, we can construct a $2$-approximation solution using dynamic programming.

Let $|T|$ be the number of time steps in the time period $T$.  Each trajectory is a piecewise-linear function that only changes directions at a time step in $T$.  Let $C_{ij}$ denote the max diameter of the $2$-approximation clustering at time $i$ over the time period $[i,j]$, $i<j$.  We can create a $|T| \times |T|$ table $\mathcal{T}$ where entry $\mathcal{T}(i, j) = C_{ij}$.  
%One clustering takes $O(\frac{1}{\epsilon}kn^2)$ and there are $|T|$ clusterings in total.  However, for each clustering, the max diameter is recalculated for each time step.  The cost of recalculating the max diameter of a clustering is $O(n/r)$.  The total number of times a clustering is recalculated is $O(n|T|/r)$.  The total time it takes to compute the table $\mathcal{T}$ is $O(n|T|^2/r + \frac{1}{\epsilon}k|T|n^2)$.

We formulate a subproblem $S(t,i)$, where $0 \leq t \leq |T|$ and $i \leq k$, for our dynamic program to find the optimal clustering of the points in the time period $[0, t]$ where there are exactly $i$ reclusterings.  Let $l(t,i)$ denote the last time step a reclustering occured for the optimal solution of $S(t,i)$.

The subproblem of our dynamic program is:
$$S(t,i) = \min( \max_{j<t}(S(j, i), C_{l(t,i)t}), \max_{j<t}(S(j, i-1), C_{tt}) )$$ 

The entry $S(t,i)$ checks $2t$ previous entries and $2t$ entries in the table $\mathcal{T}$.  The entire table takes $k|T|^2$ to execute with the additional preprocessing of the table $\mathcal{T}$.

Our lower bound proofs for static $r$-gather apply here as well.  The points arranged in any of the lower bound proofs can be static points for the duration of $T$ or may move in a fashion where the 
distances between points do not increase.  Then the arguments for static $r$-gather translate to this simple version of dynamic $r$-gather directly.

\begin{theorem}
The lower bound results for static $r$-gather apply to any definition of dynamic $r$-gather.  Further, we can approximate mobile $r$-gather, when $k$ clusterings are allowed, within a factor of $2$.
\end{theorem}

\medskip\noindent\textbf{Number of Changes in Optimal Solution.} Last, we show a lower bound on the number of changes needed if we maintain the \emph{optimal} solution at all times. We show that in this setting, the optimal clustering may change many times, as much as $O(n^3)$ times.  Consider this example: $n/2$ points lie on a line where the points are spaced apart by $1$ and $3$ points are overlapping on the ends.  In this example, $r = 3$.  The optimal clustering of the points on the line is to have three points in a row be in one cluster with a diameter of $2$.  There are three different such clusterings which differ in the parity of the clusterings.  In each clustering, there are $O(n)$ clusters.  If another point travels along the line, when it is within the boundaries of a cluster, it will just join that cluster.  However, when it reaches the boundary of a cluster and exits it, the optimal clustering would be to shift the parity of the clustering.  This results in a change in all of the clusters along the line.  The clustering change every time the point travels a distance of $2$.  Therefore, as the point travels along the line, the number of times the entire clustering changes is $\Omega(n)$ which results in a total of $\Omega(n^2)$ changes to individial clusters.  Since there are $O(n)$ points that travel along the line, the total number of clusters that change is $\Omega(n^3)$.





\subsection{Dynamic Algorithm}\label{subsec:dynamic}

In this subsection, we briefly outline the adaptation of the algorithm to mobility of nodes.

\myparagraph{Mobility management.} The challenge in maintaining the clusters in face of mobility is that motion on part of any of the nodes in a cluster requires a possible update on part of the cluster. We assume that a location service such as ~\cite{abraham04LLS} is available. This system works as follows. It divides the plane into a quadtree hierarchy, where a square region is recursively subdivided into four square subregions. Each square at at each level is assigned a location server. The presence of each node is noted at the server for squares at each level containing the node. To avoid excessive updates to the hierarchy, when a node leaves a square $s$ in level $\alpha$, the servers at level $\alpha+1$ are not updated immediately. Instead, they get updated when the node has passed out of the neighborhood of $s$ consisting of $8$ other squares in level $\alpha$. This lazy scheme guarantees a low amortized cost to keep the data up to date. 

\myparagraph{Cluster maintenance in location hierarchy.} We can adapt this scheme to our purposes as follows. Each server maintains a count of the nodes in its square. And for simplicity, we let the location servers perform the computations instead of mobile nodes and become cluster centers. Now, when a server $i$ queries for its $r$-neighborhood, this query propagates up the server hierarchy, at each level $\alpha$, checking the square $s_{\alpha}(i)$ containing $i$, and its eight neighbors, written as $N(s_{\alpha}(i))$ to see if they contain a total of $r$ mobile nodes. Suppose level $\beta$ is the first level where $N(s_{\beta}(i))$ contains $r$ nodes. The radius is this neighborhood is within a constant factor of $d_{r}(i)$. The system then returns the neighborhood $N(s_{\beta+1}(i))$ of the next higher level $\beta+1$ as the level containing at least $r$ nodes. Thus, nodes in this set plays the role of $N_{r}$ neighborhood. And the algorithms from the previous subsections apply as usual. Observe that the radius of $\beta+1$ is also $O(d_{r}(i))$. 

Next, we modify this protocol to adapt to mobility of nodes. Observe that since we take $N(s_{\beta+1}(i))$ neighborhood, a node moving from $N(s_{\beta}(i))$ to a neighboring square does not require an immediate update to $d_{r}(i)$. The update is made only when it passes out of $N(s_{\beta+1}(i))$. Thus, the number of updates caused by the mobility of a node is $O(x\log x)$ when the node has moved a distance $x$ (see~\cite{abraham04LLS}). 

The server $s_{\beta}(i)$ simply updates its nodes count on these events and does not modify cluster, until it detects that number of nodes in its neighborhood has fallen below $r$. in which case it triggers a re-clustering for all clusters with center in the neighborhood $N(s_{\beta+2}(i))$. This guarantees that cluster sizes of $r$ are preserved.

It is possible to conversely trigger re-custering when a server detects a large influx of mobile nodes. Suppose $N(s_{\beta}(i))$ is the current cluster, and  $N(s_{\alpha}(i))\subset N(s_{\beta}(i))$ detects at least $r$ nodes in its domain. Then, if $\beta-\alpha\geq 2$, it triggers a reclustering in $N(s_{\beta}(i))$.


%!TEX root = r-gather.tex

\section{Hardness}

For the application of protecting location privacy, the data points are actually in Euclidean spaces. Thus, we ask whether the hardness of approximation still holds in the Euclidean space. In the following, we assume that the input points are in the Euclidean plane.  

For the case where the diameter of a cluster is the diameter of the smallest covering disk, we show it is NP-hard to approximate better than $\sqrt{13}/2 \approx 1.802$ when $r=3$ and ${\sqrt{35}+\sqrt{3} \over 4} \approx 1.912$ when $r \geq 4$.

For the case where the diameter of a cluster is the distance between the furthest pair of points, then it is NP-hard to approximate better than $\sqrt{2+\sqrt{3}} \approx 1.931$ when $r=3$ or $4$ and $2$ when $r\geq5$.

%For the dual objective of maximizing $r$ given a fixed disk size, we can show the problem is NP-hard to approximate better than $2/3$, using instances in which the optimal $r$ value is 3. Can this be strengthened using instances whose optimal $r$ values are greater?


\begin{theorem}\label{thm:hardness1}
The $r$-gather problem for the case where the diameter of a cluster is measured by the furthest distance between two points is NP-hard to approximate better than a factor of $2$ when $r\geq5$.
\end{theorem}
\begin{proof}
Our reduction is from the NP-hard problem, planar 3SAT.  Given a formula in 3CNF composed of variables $x_i, i = 1,\dots,n$ and their complements $\overline{x_i}$, we construct an instance of $r$-gather on the plane.  Figure~\ref{fig:3satconstruction} illustrates a clause gadget of the clause $C = x_i \vee x_j \vee x_k$ and part of a variable gadget for $x_i$.  In the figure, each point represents multiple points in the same location, the number of which is noted in parenthesis.  All distances between groups of points connected by a line are distance 1 apart.  Note that all clusters shown in the figure have a diameter of 1.  If all clusters have a diameter of 1, then we can signify the parity of a variable by whether solid or dashed clusters are chosen.  Here the solid clusters signify a positive value for $x_i$ that satisfies the clause since the center point of the clause gadget is successfully assigned to a cluster.  Note that the variable gadget in Figure~\ref{fig:3satconstruction} swaps the parity of the signal sent away from the gadget.  We also include a negation gadget shown in Figure~\ref{fig:negation} that swaps the parity of the signal and can be used when connecting parts of the variable gadget together.  If an optimal solution to this $r$-gather construction can be found, the diameter of all clusters is 1.

%For the case where $r=3$ or $r = 4$, any clustering that has a cluster with diameter greater than 1 must have a cluster with diameter greater than or equal to $\sqrt{3}$.  A cluster with diameter $\sqrt{3}$ can be found in the clause gadget containing a point from each variable gadget and the center point.  There are no possible clusterings with a diameter greater than 1 or less than $\sqrt{3}$.  Therefore, it is NP-hard to approximate 3-gather and 4-gather better than a factor of $\sqrt{3}$.

The center point of the clause gadget must be assigned to a cluster that contains all $r$ points of one of the variable clusters or else a cluster of diameter 2 is forced.  WLOG, let the center point be clustered with the $r$ points of the $x_i$ gadget.  What results is the solid clusters in Figure~\ref{fig:3satconstruction} are selected above the triangle splitter and the dashed clusters are selected below the splitter.  The group of points at the top of the triangle splitter is unassigned to a cluster.  It must merge with one of the neighboring clusters which results in a cluster of diameter 2.  Therefore, it is NP-hard to approximate $r$-gather below a factor of 2 for $r\geq5$.
\end{proof}

\begin{figure}[htbp]
\begin{center}
\includegraphics[scale=.6]{figs/hardness}
\caption{Clause and splitter gadget}
\label{fig:3satconstruction}
\end{center}
\vspace{-5pt}
\end{figure}

\begin{figure}[htbp]
\begin{center}
\includegraphics[scale=.6]{figs/negation}
\caption{Signal negation gadget}
\label{fig:negation}
\end{center}
\vspace{-5pt}
\end{figure}

\begin{theorem}\label{thm:hardness2}
The $r$-gather problem for the case where the diameter of a cluster is measured by the diameter of the smallest covering disk is NP-hard to approximate better than a factor of ${\sqrt{35}+\sqrt{3} \over 4} \approx 1.912$ when $r\geq4$.
\end{theorem}
\begin{proof}
The reduction is very simlar to the proof of Theorem~\ref{thm:hardness1}.  The only difference is the splitter which is illustrated in Figure~\ref{fig:splitter}.
\end{proof}

\begin{figure}[htbp]
\begin{center}
\includegraphics[scale=.9]{figs/splitter}
\caption{Splitter gadget}
\label{fig:splitter}
\end{center}
\vspace{-5pt}
\end{figure}

\begin{corollary}\label{cor:hardness5}
The $r$-gather problem in the $L_1$ and $L_\infty$ metrics is NP-hard to approximate better than a factor of 2.
\end{corollary}

\begin{theorem}\label{thm:hardness3}
The $r$-gather problem for the case where the diameter of a cluster is the distance between the furthest pair of points, then it is NP-hard to approximate better than $\sqrt{2+\sqrt{3}} \approx 1.931$ when $r=3$ or $4$.
\end{theorem}

\begin{theorem}\label{thm:hardness4}
The $r$-gather problem for the case where the diameter of a cluster is measured by the diameter of the smallest covering disk is NP-hard to approximate better than a factor of $\sqrt{13}/2 \approx 1.802$ when $r=3$.
\end{theorem}

Corollary \ref{cor:hardness5} is consequence of Theorem \ref{thm:hardness1}.  Theorems \ref{thm:hardness3} and \ref{thm:hardness4} are proved with reductions from planar circuit SAT.  The gadgets used in the reduction are similar to the splitter gadget used in the proof of Theorem \ref{thm:hardness1}.  Details of the proofs are omitted due to space constraints.

%\begin{proof}
%We reduce from the NP-hard problem planar circuit SAT.  We are given a planar boolean circuit with a single output.  Similar to the previous proofs, a wire gadget consists of a line of points that alternate between a single point and a group of $r-1$ points at the same location.  The parity of the clusters chosen signify a true signal or a false signal.  When the clusters combine a group of $r-1$ points followed by a single point, the signal of the wire is true.  It is simple to enforce the output to be a true signal by ending the output wire with a single point.  The beginning of the input wires have a group of $r$ points so that the inputs can be either true or false.  Figure~\ref{fig:nandgadget} illustrates the NAND gadget, a universal gate.  The solid clusters illustrate two true inputs into the gate and a false output.  If either or both of the inputs is false, then two groups of points in the triangle (or all three) will become a cluster and the output will be true.  Figure~\ref{fig:splittercircuit} ilustrates the splitter circuit where the solid clusters indicate a true signal and the dashed clusters indicate a false signal.  As before, if the optimal solution to the $r$-gather construction can be found, then cluster diameter will be 1.  Otherwise, three groups will form a cluster, two from the triangle and one adjacent to the triangle.  The diameter of such a cluster is $\sqrt{13}/2 \approx 1.802$ when $r=3$.  Finally, note that in order to connect the wires, they must be able to turn somehow.   We can bend the wire such that no three groups of points can form a cluster that has diameter smaller than $\sqrt{13}/2$.  Thus concludes our proof.
%\end{proof}

%\begin{figure}[htbp]
%\begin{center}
%\includegraphics[scale=.6]{figs/nandgadget}
%\caption{NAND gadget}
%\label{fig:nandgadget}
%\end{center}
%\end{figure}

%\begin{figure}[htbp]
%\begin{center}
%\includegraphics[scale=.6]{figs/splittergadget}
%\caption{splitter gadget}
%\label{fig:splittercircuit}
%\end{center}
%\end{figure}


\pagebreak

%!TEX root = r-gather.tex

\section{Experimental Results}


We have implemented this distributed algorithm along with the $r$-gather algorithm described in~\cite{Aggarwal06achievinganonymity} to compare their clustering qualities. 
%Both were coded up in Python 2.7.12 using principally, the NetworkX, NumPy, SciPy and Matplotlib packages.

%\subsection{Data Sets}

First, in order to have an intuitive understanding of its performance in the static setting, we applied our distributed algorithm to random point clouds in the plane; Figure~\ref{fig:snapshot} shows examples of resulting clusters, for $r=3,5,7,9$, on a random point cloud of size $50$.

%But in the other simulations we used all $9386$ trajectories.
%The points shown are a snapshot of the GPS co-ordinates of 1500 cars driving around Shenzhen city in China.

\begin{figure*}[htpb]
\begin{center}
\begin{tabular}{cc}
\vspace*{-8mm}
	\includegraphics[scale=0.25]{figs/r3.png} &
	\includegraphics[scale=0.25]{figs/r5.png} \\
\vspace*{-12mm}
%\footnotesize (i) & \footnotesize (ii) \\
	\includegraphics[scale=0.25]{figs/r7.png} & 
	\includegraphics[scale=0.25]{figs/r9.png} 	\\
\vspace*{-6mm}
% \footnotesize (iii)& \footnotesize (iv) \\
\end{tabular}
\end{center}
%\vspace*{-6mm}
	\caption{\footnotesize The clusters produced by the distributed algorithm for a set of $50$ points uniformly randomly distributed, with $r=3$ in (top left), $r=5$ in (top right), $r=7$ in (bottom left) and $r=9$ in (bottom right) respectively. }
	\label{fig:snapshot}
%\vspace{-0.3in}
\end{figure*}

%\begin{figure}[h]
%\begin{center}
%\includegraphics[width=3in]{figs/r3.png}
%\end{center}
%\end{figure}
%
%
%\begin{figure}[h]
%\begin{center}
%\includegraphics[width=3in]{figs/r5.png}
%\end{center}
%\end{figure}
%
%
%\begin{figure}[h]
%\begin{center}
%\includegraphics[width=3in]{figs/r7.png}
%\end{center}
%\end{figure}
%
%
%\begin{figure}[h]
%\begin{center}
%\includegraphics[width=3in]{figs/r9.png}
%\end{center}
%\end{figure}




%The $X$ and $Y$ axes in the figures above respectively denote longitude and latitude. 
%For generating these clusters and other experiments on the distributed algorithm to be described next, latitude and longitude pairs were treated simply as $x,y$ co-ordinates in $\mathbb{R}^2$. 

%\vspace{3mm}


%\subsection{Experiments}

%\vspace{3mm}


%\subsection{ Experimental Setup }        
%\vspace{2mm}


To test the quality of the clusterings generated for these algorithms for different choices of $r$, we calculated the maximum of the clusters' diameters. Our distributed algorithm has many variations depending on how exactly certain steps are performed; e.g., we can vary how one selects a good maximal independent set among the $r$-neighborhoods, $N_r(p_i)$, of the nodes $p_i\in P$. In our first implementation, we do this as follows: We compute the distance $d_r(p_i)$ from $p_i$ to its $r$th nearest neighbor, for each $p_i$; we then add to the independent set that $r$-neighborhood $N_r(p_i)$ that minimizes $d_r(p_i)$, remove the points $p_i$ and $N_r(p_i)$ from $P$, and then repeat this selection process, until no $r$-neighborhoods remain.  In another variant of our implementation, we select the maximal independent set of $r$-neighborhoods greedily (with preference to those of smallest diameter); in fact, we repeat the process $K$ times (by default, we chose $K=20$), and select the best solution from the multiple runs.
%% Joe:  Question for Gaurish:  why repeat?  Is there an element of random choice in the greedy algorithm? If so, we need to clarify this!  Also, what does ``best'' mean? (the one whose largest diameter is smallest, I assume?)

% Use to blank out text portions that are old, but not discarded                                                                                  \newcommand{\old}[1]{{}}
\old{
Given a graph G=(V,E), find a maximal set in G.
0. initialize I = empty.
1. pick a node, v, in V at random; place it in I
2. remove v and its neighbors from V.
3. if V is nonempty, go to 1

It this right?
(I just looked at the source code:
https://networkx.github.io/documentation/networkx-1.9.1/_modules/networkx/algorithms/mis.html#maximal_independent_set   )

We should make this more clear in the text, of course.

There are many other ways one can imagine to find a maximal indep set.

A very natural greedy is this:  Pick a lowest degree node v in V to add to I (then remove it, and its neighbors, repeat).
We need a rule to break ties for lowest degree:  in our setting, we could do so in favor of smallest d_r(v).

Note too, that instead of using d_r(v) as a measure of "size" (radius, or diameter), we could use the actual diameter of
the r-neighborhood, N_r(v).

I would suggest that we try out these other variants.....
}

For each clustering algorithm, we calculate the following two statistics:
\bitem
\item  The maximum, over all clusters, of the diameters of the clusters.
\item  The $90$th percentile of the diameters of the clusters.
%\item  The  maximum over all clusters, the $90$th percentile of point-distances within a cluster.
\eitem
We also plot the value of $d_r^{\max}=\max_i d_r(p_i)$, which we know (Observation~\ref{obs:lower-bound}) is a lower bound on the optimal diameter, $D_{OPT}$, in an $r$-gather solution.

We ran our algorithms on a data set of the GPS coordinates of $9386$
taxicabs in Shenzhen (China), sampled at 5 minute intervals over one
full day.
Figure~\ref{fig:comparison} shows the results of our algorithm, in comparison
to~\cite{Aggarwal06achievinganonymity} and $d_{r}^{\max}$ as a baseline, on a snapshot containing $60$ random users from the dataset. Figure~\ref{fig:comparison-90} shows the $90\%$ percentile results.  

\begin{figure}[h]
\begin{center}
\includegraphics[width=3in]{figs/figure_1.png}
\caption{Maximum cluster diameter. Blue: approximation algorithm from~\cite{Aggarwal06achievinganonymity}; Red: distributed algorithm; Green: $d_{r}^{\max}$.}\label{fig:comparison}
\end{center}
\end{figure}


\begin{figure}[h]
\begin{center}
\includegraphics[width=3in]{figs/90thpercentile_60Cars.png}
\caption{$90\%$ percentile of all cluster diameter. Blue: approximation algorithm from~\cite{Aggarwal06achievinganonymity}; Red: distributed algorithm; Green: $d_{r}^{\max}$; Magenta: the distributed algorithm with the best maximal independent set run over $20$ iterations.}\label{fig:comparison-90}
\end{center}
\end{figure}

Figure~\ref{fig:large} shows results on a larger dataset of $1500$
mobile users; the distributed algorithm still performs well. Figure~\ref{fig:large-90} shows the $90\%$ percentile results.  


\begin{figure}[h]
\begin{center}
\includegraphics[width=3in]{figs/cars1500_4Approx.png}
\caption{Maximum cluster diameter on a snapshot of 1500 mobile users. Red: distributed algorithm; Green: $d_{r}^{\max}$; Magenta: the distributed algorithm with the best maximal independent set run over $20$ iterations.}\label{fig:large}
\end{center}
\end{figure}

\begin{figure}[h]
\begin{center}
\includegraphics[width=3in]{figs/90thpercentile_1500cars.png}
\caption{$90\%$ percentile of all cluster diameter. 
Red: distributed algorithm; Green: $d_{r}^{\max}$; Magenta: the distributed algorithm with the best maximal independent set run over $20$ iterations.}\label{fig:large-90}
\end{center}
\end{figure}

%\vspace{3mm}

%\bitem
%\item  The maximum over all clusters, the diameter of a cluster.  
%\item  The maximum over all clusters, the $90$th percentile of point-distances within a cluster.
%\eitem
%\vspace{3mm}

%\vspace{3mm}




%\vspace{20mm}




%We implemented the distributed algorithm and compared it with~ \cite{Aggarwal06achievinganonymity} and with the lower bound of $d_{r}^{\max}$  on real location data from a trajectory dataset of 9000 mobile users in Shenzen city in china. 

%
%\begin{figure}[h]
%\begin{center}
%\includegraphics[width=3in]{figs/figure_1.png}
%\caption{Max cluster diameter. Black curve: approximation algorithm from~\cite{Aggarwal06achievinganonymity};
%  Red curve: distributed algorithm; Green curve: $d_{r}^{\max}$.}\label{fig:comparison}
%\end{center}
%\end{figure}






%We implemented the distributed algorithm and compared it with~\cite{Aggarwal06achievinganonymity} and with the lower bound of $d_{r}^{\max}$ on real location data from a trajectory dataset of 9000 mobile users in Shenzen city in china. 
Our main observations are:

\begin{itemize}
\item Our distributed algorithm usually produces better results than the $2$-approximation algorithm of~\cite{Aggarwal06achievinganonymity} in practice, although the approximation bound for the distributed algorithm is worse in theory.
\item The distributed algorithm runs faster and therefore can be run on larger datasets
\item The results (maximum cluster diameters) are close to the lower bound of $d_{r}^{\max}$.
\end{itemize}


%
%\begin{figure}[h]
%\begin{center}
%\includegraphics[width=3in]{figs/figure_1.png}
%\caption{Max cluster diameter. Black curve: approximation algorithm from~\cite{Aggarwal06achievinganonymity};
%  Red curve: distributed algorithm; Green curve: $d_{r}^{\max}$.}\label{fig:comparison}
%\end{center}
%\end{figure}


%\vspace*{-3mm}

\section{Conclusion}
In this paper we investigated the $r$-gather problem, a variant of geometric clustering for the mobile settings. We improved hardness results for metrics in the Euclidean setting, and proposed algorithm for the dynamic setting when nodes move around and regrouping is allowed. Further, we proposed a distributed algorithm which uses local operations that gives a $4$-approximation. We evaluated the algorithms on a real data set and show that actually the distributed algorithm (though with worst case approximation ratio worse than previous centralized algorithms) actually perform better in practice, in terms of maximum cluster diameter. We expect that the algorithms find other applications in mobile computing. 



%\begin{small}
%\bibliographystyle{abbrv}
\bibliographystyle{ACM-Reference-Format}
\bibliography{r-gather,privacy,jie}
%\end{small}


\end{document}
